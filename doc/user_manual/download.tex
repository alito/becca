\chapter{Get and run Becca}

Each chapter in this guide is designed to help you do something specific with Becca. This chapter helps you to get a copy of Becca on your local machine and run it on some generic worlds.

Becca can be downloaded from www.openbecca.org or www.sandia.gov/rohrer. The latest bleeding edge version (for which any of this documentation may already be outdated) can be downloaded from the github repository at

https://github.com/matt2000/becca 

Becca is intended to be runnable on any hardware platform. It relies on Python, NumPy, and Matplotlib. Becca 0.4.0 was developed on Python 2.7, NumPy 1.6.1, and matplotlib 1.2. The code makes use of at least one NumPy call, count\_nonzero(), that is not supported in NumPy 1.5.x. Becca has been run several different platforms (Mac OS 10.6.8, Ubuntu 12.04 \footnote{From developer SeH: BECCA's dependency on NumPy 1.6 is provided by the latest Ubuntu 12.04 package (but not Ubuntu 11.10 which provides NumPy 1.5).  A note on the website indicating this might be helpful to some users.  Matplotlib is also provided, so everything seems to work fine on Ubuntu 12.04.}) and IDEs (Eclipse, IDLE)\footnote{Any notes on successes or incompatibilities would be very welcome at openbecca.org.}. 

The benchmark.py module automatically runs Becca on a collections of worlds that are included with the download. Run it in a Python interpreter to get a report of Becca's performance in the worlds. Benchmark.py can be used both to compare Becca's speed on different computing platforms and, more importantly, to compare different variants of Becca against each other.

The worlds in benchmark.py are intended to be simple, but to test Becca's fundamental learning capabilities. Becca is an agent, in the sense that it makes decisions in order to achieve a goal, but it is intended for use in many different settings, each of which is referred to as a world. The worlds tested in benchmark.py include one and two dimensional grid worlds and one and two dimensional visual worlds. The reward provided by each world gives motivation to Becca to behave in certain ways. When it behaves correctly, it maximizes its reward. This is Becca's one and only goal.