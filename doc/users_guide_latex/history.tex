\chapter{Revision history}

\section*{Version 0.4.3, November 9, 2012}

\begin{itemize}
\item {\bf Major change} Groups are eliminated. Features are created directly now using the old group-creation algorithm. This in turn spawned a large number of minor changes to support it. The end result is that features in the higher-levels of the hierarchy tend to be created more gracefully, make more sense to a human user, and naturally terminate. The number of user-selected parameters in \textsc{Becca} also went down a bit.
\item The maximum possible size for large variables (the co-activity estimate and the model) is allocated on startup. This helps prevent running out of memory mid-run.
\item The \texttt{learner} was renamed \texttt{actor} to more accurately reflect its function and to provide symmetry with \texttt{perceiver} in the sense of ``perception-action loops". 
\end{itemize}

\section*{Version 0.4.2, September 28, 2012}

\begin{itemize}
\item A chapter describing the operation of the reinforcement learner has been added to the Users Guide.
\item The image-based tasks all implement center-surround filtering as a preprocessing step.
\item Perceiver parameters have been adjusted to produce more intuitive features in the image-based tasks.
\item Features in the image-based tasks are displayed as receptive fields, resulting in a higher-contrast format.
\item Mutual co-activity has been modified to be the minimum, rather than the product, of two co-activities.
\end{itemize}


\section*{Version 0.4.1, August 17, 2012}

\begin{itemize}
\item Nearly all numpy 1D arrays were converted to 2D arrays for consistency   
\item Co-activity has been modified to be symmetric.
\item Model transitions' effects are updated with each observation or use. That is, they are updated both when they are observed and when they are used by planner as a basis for selecting an action.
\item Model transitions now include an effect uncertainty and a reward uncertainty. 
\item All the features in a group are now created when the group is created. They evolve over time based on the inputs observed.
\item Feature activity is driven both by excitation and fatigue.
\item Feature evolution is driven both by activity and inhibition.
\end{itemize}


\section*{Version 0.4.0, June 8, 2012}

\begin{itemize}
\item Ported to Python 2.7 from MATLAB. Props to to Alejandro Dubrovsky.
\item Agent and World objects disentangled.
\item Grouper object expanded to be responsible for all aspects of feature creation.
\item Learner object created, responsible for all model building and learning.
\item State object created, containing sensors, primitives, actions, and features.
\item Model structure revised to reflect new state structure.
\item Deliberate actions are now attended immediately.
\item Benchmark module added for measuring performance on all worlds in the battery.
\item Becca now works on all battery worlds, but with much room for improvement.
\item Users Guide added. 
\end{itemize}

